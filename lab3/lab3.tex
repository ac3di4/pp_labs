\documentclass[a4paper, 12pt]{article}

\usepackage{prelab}

\usepackage{graphicx}
\usepackage{pgfplots}
\usepackage{pgfplotstable}
\usepackage{titlesec}
\usepackage{listings}

\pgfplotsset{width=12cm, compat=1.18}

\begin{document}

\prefixlab{3}{Реализация алгоритма с использованием технологии \textit{OpenMP}}

\section{Сортировка Шелла}

\subsection{Принцип работы}

При сортировке Шелла сначала сравниваются и сортируются между собой значения, стоящие один от другого на некотором расстоянии $d$.
После этого процедура повторяется для некоторых меньших значений $d$, а завершается сортировка Шелла упорядочиванием элементов при $d = 1$ (то есть обычной сортировкой вставками).
Эффективность сортировки Шелла в определённых случаях обеспечивается тем, что элементы «быстрее» встают на свои места (в простых методах сортировки, например, пузырьковой, каждая перестановка двух элементов уменьшает количество инверсий в списке максимум на 1, а при сортировке Шелла это число может быть больше).

Для определённости будет рассматриваться классический вариант, когда изначально $d = \frac{n}{2}$ и уменьшается по закону $d_{i+1} = \frac{d_{i}}{2}$, пока не достигнет $1$. 
Здесь $n$ обозначает длину сортируемого массива.

Тогда в худшем случае сортировка займет $O(n^2)$.

Блок схема сортировки Шелла:\\
\includegraphics[scale=0.6]{res/original.png}

\subsection{Параллелизация}

Как и в предыдущих лабораторных, в первую очередь следует попробовать сделать параллельным цикл.

Задаем число потоков и общие переменные через \textit{omp parallel}.
Однозначно общими должны быть массив и его длинна.

Так как внутренний цикл по i по сути затрагивает только $d$-e элементы относительно $i$-го, то:

\begin{lstlisting}[language=C, basicstyle=\scriptsize]
 #pragma omp parallel num_threads(THREADS) shared(array, count) default(none)
    for (int d = count / 2; d > 0; d /= 2) {
        const int cd = d;
        #pragma omp for
        for (int i = cd; i < count; ++i) {
            for (int j = i - cd; j >= 0 && array[j] > array[j + cd]; j -= cd) {
                int temp = array[j];
                array[j] = array[j + cd];
                array[j + cd] = temp;
            }
        }
    }
\end{lstlisting}

Здесь так же видно, что $d$ вынесена в константу $cd$.
Это сделано для того, что бы \textit{OpenMP} не принял меры предосторожности в цикле по $i$.
Он может это сделать так как $d$ меняется во внешнем цикле, но он не знает меняется ли во внутреннем.

\section{Экспериментальные данные}

На каждое число потоков отводилось 20 запусков.

\subsection{Время выполнения}

Для начала я решил взглянуть не только на среднюю скорость выполнения, но и на крайние варианты:

\vspace{0.5cm}

\begin{tikzpicture}
    \begin{axis}[
        xlabel={Число потоков},
        ylabel={Время (мс)},
        legend pos=north east,
    ]
    \addplot table [x index = 0, y index = 1, col sep=comma] {data/base.csv};
    \addplot table [x index = 0, y index = 2, col sep=comma] {data/base.csv};
    \addplot table [x index = 0, y index = 3, col sep=comma] {data/base.csv};
    \legend{Худшее время, Лучшее время, Среднее время}
    \end{axis}
\end{tikzpicture}

\vspace{0.5cm}

Крайне заметно, что многопоточная программа работает куда быстрее её конкурентов.
Единственным исключением является запуск при 2-x потоках.

Аналогично первой лабораторной рассмотрим теперь данные с оптимизацией:

\vspace{0.5cm}

\begin{tikzpicture}
    \begin{axis}[
        xlabel={Число потоков},
        ylabel={Время (мс)},
        legend pos=north east,
    ]
    \addplot table [x index = 0, y index = 1, col sep=comma] {data/opt.csv};
    \addplot table [x index = 0, y index = 2, col sep=comma] {data/opt.csv};
    \addplot table [x index = 0, y index = 3, col sep=comma] {data/opt.csv};
    \legend{Худшее время, Лучшее время, Среднее время}
    \end{axis}
\end{tikzpicture}

\vspace{0.5cm}

Как видно на графике выше, повышение числа потоков уменьшает среднее время исполнения.

\subsection{Прирост производительности}

В целом с увеличением числа потоков производительность растет.
Рассмотрим ускорение многопоточной программы относительно однопоточной.
Для не оптимизированной сборки:

\vspace{0.5cm}

\begin{tikzpicture}
    \begin{axis}[
        xlabel={Число потоков},
        ylabel={Прирост (\%)},
        ybar interval=0.7,
    ]
    \addplot table [x index = 0, y index = 1, col sep=comma] {data/cmp_base.csv};
    \end{axis}
\end{tikzpicture}

\vspace{0.5cm}

Для оптимизированной сборки:

\vspace{0.5cm}

\begin{tikzpicture}
    \begin{axis}[
        xlabel={Число потоков},
        ylabel={Прирост (\%)},
        ybar interval=0.7,
    ]
    \addplot table [x index = 0, y index = 1, col sep=comma] {data/cmp_opt.csv};
    \end{axis}
\end{tikzpicture}

\vspace{0.5cm}


\section{Заключение}

В данной работе было исследовано ускорение, получаемое при использовании нескольких потоков в задании о сортировке массива сортировкой Шелла.
Была усовершенствована предоставленная программа и собранны данные.
Так же был написан скрипт, подсчитывающий прирост производительности относительно одного потока.
Оформлен отчет.

В ходе работы было выяснено, что в применение нескольких потоков крайне положительно влияет на итоговую производительность.
Из 30 многопоточных сборок только 2 превосходили обычную менее чем в 2 раза.
При этом наблюдался прирост вплоть до 5-ти раз.

\appendix

\titleformat{\section}[display]
  {\normalfont\Large\bfseries}
  {\centering Приложение\ \thesection\\}
  {0pt}{\Large\centering}
\renewcommand{\thesection}{\Asbuk{section}}

\section{Использованные программные коды}

Для проверки версии \textit{OpenMP} использовался следующий код:
\lstinputlisting[language=C, basicstyle=\small]{code/omp_ver.c}
\vspace{0.5cm}

Для измерения времени исполнения алгоритма использовался следующий код (выводит \textit{csv} в стандартный вывод):
\lstinputlisting[language=C, basicstyle=\tiny]{code/main.c}
\vspace{0.5cm}

Для вычисления эффективности многопоточной программы по отношению к однопоточной использовался следующий скрипт:
\lstinputlisting[language=Python, basicstyle=\small]{code/compare.py}
\vspace{0.5cm}

\section{Таблицы c практическими результатами}

Таблица без оптимизаций:

\vspace{0.3cm}

\pgfplotstabletypeset[
 col sep=comma,
 columns={Threads,Worst (ms),Best (ms),Avg (ms)},
]{data/base.csv}

\vspace{0.5cm}

Таблица с оптимизациями:

\vspace{0.3cm}

\pgfplotstabletypeset[
 col sep=comma,
 columns={Threads,Worst (ms),Best (ms),Avg (ms)},
]{data/opt.csv}

\vspace{0.5cm}

Таблица сравнений без оптимизаций:

\vspace{0.3cm}

\pgfplotstabletypeset[
 col sep=comma,
 columns={Threads,Efficiency},
]{data/cmp_base.csv}

\vspace{0.5cm}

Таблица сравнений c оптимизациями:

\vspace{0.3cm}

\pgfplotstabletypeset[
 col sep=comma,
 columns={Threads,Efficiency},
]{data/cmp_opt.csv}


\end{document}