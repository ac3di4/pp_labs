\documentclass[a4paper, 12pt]{article}

\usepackage{prelab}

\usepackage{graphicx}
\usepackage{pgfplots}
\usepackage{pgfplotstable}
\usepackage{titlesec}
\usepackage{listings}

\pgfplotsset{width=12cm, compat=1.18}

\begin{document}

\prefixlab{3}{Реализация алгоритма с использованием технологии \textit{OpenMP}}

\section{Реализация сортировки Шелла с использованием \textit{MPI}}

\subsection{Параллелизация алгоритма}

Цикл по различным $d$ будет выполнятся в главном процессе.
При этом на каждой своей итерации он будет рассылать фрагменты массива другим потокам.
Есть два способа делать это:
\begin{itemize}
    \item Брать последовательные фрагменты массива;
    \item Брать $d$-е элементы относительно i.
\end{itemize}

\subsection{Последовательные фрагменты}

Суть данного способа -- разделить массив на (почти) равные части между процессами и произвести сортировку на каждом.
У данного способа есть 2 существенные проблемы:
\begin{enumerate}
    \item Размер массива может ровно не делиться на количество процессов;
    \item Фрагменты массива при получении главным потоком нужно дополнительно отсортировать. Простейшая сортировка слиянием.
\end{enumerate}

Первая проблема нивелируется тем, что если число процессов невелико, то оставшиеся элементы можно отдать одному из них не потеряв сильно в производительности.

Вторая проблема гораздо существеннее. 
После параллельной сортировки сами фрагменты отсортированы, но не относительно друг-друга.
В качестве решения можно произвести сортировку слиянием.

Её тоже можно сделать параллельно, но я затрудняюсь это реализовать.

\subsection{$d$-е элементы}

С помощью векторного типа \textit{MPI} можно передавать $d$-е элементы.
Тогда сортировка в под-процессах превращается в обычную сортировку слиянием.

Однако этот подход так же не лишен проблем:
\begin{itemize}
    \item Количество и длинна под-векторов, на которые необходимо разбивать меняется каждую итерацию внешнего цикла по $d$ из-за чего произвести равномерное распределение сложнее;
    \item Если одному процессу передается несколько под-векторов, то не ясно в каком порядке он их вернет.
\end{itemize}

Вторую проблему можно решить путем задания определенного порядка отправления и возврата.

Первая проблема даже при хорошем распределении останется проблемой.
К тому же пересылка большого объема данных звучит не очень хорошо.

\subsection{Выбранный алгоритм}

Учитывая вышеописанные минусы различных подходов я решил остановится на первом методе.

По своей сути второй метод совершает больше пересылок, когда как первый не использует все процессы в конце.

\subsection{Исходный код}

Полный исходный код предоставлен в приложении А.
\section{Экспериментальные данные}

На каждое число потоков отводилось 20 запусков.

\subsection{Время выполнения}

Для начала я решил взглянуть не только на среднюю скорость выполнения, но и на крайние варианты:

\vspace{0.5cm}

\begin{tikzpicture}
    \begin{axis}[
        xlabel={Число потоков},
        ylabel={Время (мс)},
        legend pos=north east,
    ]
    \addplot table [x index = 0, y index = 1, col sep=comma] {data/base.csv};
    \addplot table [x index = 0, y index = 2, col sep=comma] {data/base.csv};
    \addplot table [x index = 0, y index = 3, col sep=comma] {data/base.csv};
    \legend{Худшее время, Лучшее время, Среднее время}
    \end{axis}
\end{tikzpicture}

\vspace{0.5cm}

Крайне заметно, что многопоточная программа работает куда быстрее её конкурентов.
Единственным исключением является запуск при 2-x потоках.

Аналогично первой лабораторной рассмотрим теперь данные с оптимизацией:

\vspace{0.5cm}

\begin{tikzpicture}
    \begin{axis}[
        xlabel={Число потоков},
        ylabel={Время (мс)},
        legend pos=north east,
    ]
    \addplot table [x index = 0, y index = 1, col sep=comma] {data/opt.csv};
    \addplot table [x index = 0, y index = 2, col sep=comma] {data/opt.csv};
    \addplot table [x index = 0, y index = 3, col sep=comma] {data/opt.csv};
    \legend{Худшее время, Лучшее время, Среднее время}
    \end{axis}
\end{tikzpicture}

\vspace{0.5cm}

Как видно на графике выше, повышение числа потоков уменьшает среднее время исполнения.

\subsection{Прирост производительности}

В целом с увеличением числа потоков производительность растет.
Рассмотрим ускорение многопоточной программы относительно однопоточной.
Для не оптимизированной сборки:

\vspace{0.5cm}

\begin{tikzpicture}
    \begin{axis}[
        xlabel={Число потоков},
        ylabel={Прирост (\%)},
        ybar interval=0.7,
    ]
    \addplot table [x index = 0, y index = 1, col sep=comma] {data/cmp_base.csv};
    \end{axis}
\end{tikzpicture}

\vspace{0.5cm}

Для оптимизированной сборки:

\vspace{0.5cm}

\begin{tikzpicture}
    \begin{axis}[
        xlabel={Число потоков},
        ylabel={Прирост (\%)},
        ybar interval=0.7,
    ]
    \addplot table [x index = 0, y index = 1, col sep=comma] {data/cmp_opt.csv};
    \end{axis}
\end{tikzpicture}

\vspace{0.5cm}


\section{Заключение}

В данной работе было исследовано ускорение, получаемое при использовании нескольких потоков в задании о сортировке массива сортировкой Шелла.
Была усовершенствована предоставленная программа и собранны данные.
Так же был написан скрипт, подсчитывающий прирост производительности относительно одного потока.
Оформлен отчет.

В ходе работы было выяснено, что в применение нескольких потоков крайне положительно влияет на итоговую производительность.
Из 30 многопоточных сборок только 2 превосходили обычную менее чем в 2 раза.
При этом наблюдался прирост вплоть до 5-ти раз.

\appendix

\titleformat{\section}[display]
  {\normalfont\Large\bfseries}
  {\centering Приложение\ \thesection\\}
  {0pt}{\Large\centering}
\renewcommand{\thesection}{\Asbuk{section}}

\section{Использованные программные коды}

Для проверки версии \textit{OpenMP} использовался следующий код:
\lstinputlisting[language=C, basicstyle=\small]{code/omp_ver.c}
\vspace{0.5cm}

Код, использовавшийся для проверки функциональности \textit{MPI}
\lstinputlisting[language=C, basicstyle=\tiny]{code/original.c}
\vspace{0.5cm}

Для измерения времени исполнения программы с использованием \textit{OpenMP} использовался следующий код(выводит \textit{csv} в стандартный вывод):
\lstinputlisting[language=C, basicstyle=\tiny]{code/omp_main.c}
\vspace{0.5cm}

Для измерения времени исполнения программы с использованием \textit{MPI} использовался следующий код(выводит \textit{csv} в стандартный вывод):
\lstinputlisting[language=C, basicstyle=\tiny]{code/omp_main.c}
\vspace{0.5cm}

А так же для этой цели использовался скрипт:
\lstinputlisting[language=Python, basicstyle=\tiny]{code/runmpi.py}
\vspace{0.5cm}

\section{Таблицы c практическими результатами}

\textit{OpenMP}:

\vspace{0.3cm}

\pgfplotstabletypeset[
 col sep=comma,
 columns={Threads,Worst (ms),Best (ms),Avg (ms)},
]{data/omp.csv}

\vspace{0.5cm}

\textit{MPI}:

\vspace{0.3cm}

\pgfplotstabletypeset[
 col sep=comma,
 columns={Threads,Worst (ms),Best (ms),Avg (ms)},
]{data/mpi.csv}

\vspace{0.5cm}


\end{document}