\documentclass[a4paper, 12pt]{article}

\usepackage{prelab}

\usepackage{graphicx}
\usepackage{pgfplots}
\usepackage{pgfplotstable}
\usepackage{titlesec}
\usepackage{listings}

\pgfplotsset{width=12cm, compat=1.18}

\begin{document}

\prefixlab{5}{Технология \textit{MPI}. Введение}

\section{Работа с \textit{MPI}}

Стоит отметить, что в используемой мной среде для компиляции программ, поддерживающих \textit{MPI} обязательно использование специального компилятора \textit{mpicc}.

Вывод программы в однопоточном режиме:
\begin{tiny}
\begin{verbatim}
MPI Comm Size: 1;
MPI Comm Rank: 0;
Processor #0 has array: 788159773 2052308573 1377030627 1699618045 676203154 299802456 1767965774 1838448927 1686836254 1335355396
Processor #0 checks items 0 .. 9;
Processor #0 reports local max = 2052308573;

*** Global Maximum is 2052308573;
MPI Finalize returned (0);
\end{verbatim}
\end{tiny}

Вывод программы в многопоточном режиме при запуске с 4-мя процессами:
\begin{tiny}
\begin{verbatim}
MPI Comm Size: 4;
MPI Comm Rank: 2;
MPI Comm Size: 4;
MPI Comm Rank: 3;
MPI Comm Size: 4;
MPI Comm Rank: 0;
Processor #0 has array: 788159773 2052308573 1377030627 1699618045 676203154 299802456 1767965774 1838448927 1686836254 1335355396
Processor #0 checks items 0 .. 1;
Processor #0 reports local max = 2052308573;
Processor #3 has array: 788159773 2052308573 1377030627 1699618045 676203154 299802456 1767965774 1838448927 1686836254 1335355396
Processor #3 checks items 7 .. 9;
Processor #3 reports local max = 1838448927;
MPI Comm Size: 4;
MPI Comm Rank: 1;
Processor #1 has array: 788159773 2052308573 1377030627 1699618045 676203154 299802456 1767965774 1838448927 1686836254 1335355396
Processor #1 checks items 2 .. 4;
Processor #1 reports local max = 1699618045;
Processor #2 has array: 788159773 2052308573 1377030627 1699618045 676203154 299802456 1767965774 1838448927 1686836254 1335355396
Processor #2 checks items 5 .. 6;
Processor #2 reports local max = 1767965774;
MPI Finalize returned (0);

*** Global Maximum is 2052308573;
MPI Finalize returned (0);
MPI Finalize returned (0);
MPI Finalize returned (0);
\end{verbatim}
\end{tiny}

\section{Экспериментальные данные}

В программах использовалось до 6 потоков/процессов и 10 запусков на поток/процесс.

\subsection{Результаты выполнения}

Рассмотрим результаты выполнения \textit{MPI}:

\vspace{0.5cm}

\begin{tikzpicture}
    \begin{axis}[
        xlabel={Число процессов},
        ylabel={Время (мс)},
        legend pos=north east,
    ]
    \addplot table [x index = 0, y index = 1, col sep=comma] {data/mpi.csv};
    \addplot table [x index = 0, y index = 2, col sep=comma] {data/mpi.csv};
    \addplot table [x index = 0, y index = 3, col sep=comma] {data/mpi.csv};
    \legend{Худшее время, Лучшее время, Среднее время}
    \end{axis}
\end{tikzpicture}

\vspace{0.5cm}

Так же для чистоты эксперимента я решил обновить данные первой лабораторной работы:

\vspace{0.5cm}

\begin{tikzpicture}
    \begin{axis}[
        xlabel={Число потоков},
        ylabel={Время (мс)},
        legend pos=north east,
    ]
    \addplot table [x index = 0, y index = 1, col sep=comma] {data/omp.csv};
    \addplot table [x index = 0, y index = 2, col sep=comma] {data/omp.csv};
    \addplot table [x index = 0, y index = 3, col sep=comma] {data/omp.csv};
    \legend{Худшее время, Лучшее время, Среднее время}
    \end{axis}
\end{tikzpicture}

\vspace{0.5cm}

В отличии от предыдущей лабораторной положительные тенденции заметны не только у \textit{OpenMP}:

\vspace{0.5cm}

\begin{tikzpicture}
    \begin{axis}[
        xlabel={Число потоков/процессов},
        ylabel={Время (мс)},
        legend pos=north east,
    ]
    \addplot table [x index = 0, y index = 3, col sep=comma] {data/omp.csv};
    \addplot table [x index = 0, y index = 3, col sep=comma] {data/mpi.csv};
    \legend{\textit{OpenMP}, \textit{MPI}}
    \end{axis}
\end{tikzpicture}

\vspace{0.5cm}

При 6-ти потоках их производительность в данной задаче можно считать одинаковой.
Однако разработка программы на \textit{MPI} заняла гораздо больше времени и ресурсов.


\section{Заключение}

В данной работе было исследовано ускорение, получаемое при использовании технологии \textit{MPI} в задании о поиске максимума.
Была усовершенствована предоставленная программа и собраны данные.
Так же был написан скрипт для сбора данных \textit{MPI}.
Оформлен отчет.

В ходе работы было выяснено, что применение \textit{MPI} в данной задаче негативно сказывается на временных показателях.
Могу предположить, что причиной является затратная по времени операция пересылки массива.

\appendix

\titleformat{\section}[display]
  {\normalfont\Large\bfseries}
  {\centering Приложение\ \thesection\\}
  {0pt}{\Large\centering}
\renewcommand{\thesection}{\Asbuk{section}}

\section{Использованные программные коды}

Для проверки версии \textit{OpenMP} использовался следующий код:
\lstinputlisting[language=C, basicstyle=\small]{code/omp_ver.c}
\vspace{0.5cm}

Код, использовавшийся для проверки функциональности \textit{MPI}
\lstinputlisting[language=C, basicstyle=\tiny]{code/original.c}
\vspace{0.5cm}

Для измерения времени исполнения программы с использованием \textit{OpenMP} использовался следующий код(выводит \textit{csv} в стандартный вывод):
\lstinputlisting[language=C, basicstyle=\tiny]{code/omp_main.c}
\vspace{0.5cm}

Для измерения времени исполнения программы с использованием \textit{MPI} использовался следующий код(выводит \textit{csv} в стандартный вывод):
\lstinputlisting[language=C, basicstyle=\tiny]{code/omp_main.c}
\vspace{0.5cm}

А так же для этой цели использовался скрипт:
\lstinputlisting[language=Python, basicstyle=\tiny]{code/runmpi.py}
\vspace{0.5cm}

\section{Таблицы c практическими результатами}

\textit{OpenMP}:

\vspace{0.3cm}

\pgfplotstabletypeset[
 col sep=comma,
 columns={Threads,Worst (ms),Best (ms),Avg (ms)},
]{data/omp.csv}

\vspace{0.5cm}

\textit{MPI}:

\vspace{0.3cm}

\pgfplotstabletypeset[
 col sep=comma,
 columns={Threads,Worst (ms),Best (ms),Avg (ms)},
]{data/mpi.csv}

\vspace{0.5cm}


\end{document}
