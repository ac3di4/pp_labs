\section{Экспериментальные данные}

В программах использовалось до 6 потоков/процессов и 10 запусков на поток/процесс.

\subsection{Результаты выполнения}

Рассмотрим результаты выполнения \textit{MPI}:

\vspace{0.5cm}

\begin{tikzpicture}
    \begin{axis}[
        xlabel={Число процессов},
        ylabel={Время (мс)},
        legend pos=north east,
    ]
    \addplot table [x index = 0, y index = 1, col sep=comma] {data/mpi.csv};
    \addplot table [x index = 0, y index = 2, col sep=comma] {data/mpi.csv};
    \addplot table [x index = 0, y index = 3, col sep=comma] {data/mpi.csv};
    \legend{Худшее время, Лучшее время, Среднее время}
    \end{axis}
\end{tikzpicture}

\vspace{0.5cm}

Так же для чистоты эксперимента я решил обновить данные первой лабораторной работы:

\vspace{0.5cm}

\begin{tikzpicture}
    \begin{axis}[
        xlabel={Число потоков},
        ylabel={Время (мс)},
        legend pos=north east,
    ]
    \addplot table [x index = 0, y index = 1, col sep=comma] {data/omp.csv};
    \addplot table [x index = 0, y index = 2, col sep=comma] {data/omp.csv};
    \addplot table [x index = 0, y index = 3, col sep=comma] {data/omp.csv};
    \legend{Худшее время, Лучшее время, Среднее время}
    \end{axis}
\end{tikzpicture}

\vspace{0.5cm}

В отличии от предыдущей лабораторной положительные тенденции заметны не только у \textit{OpenMP}:

\vspace{0.5cm}

\begin{tikzpicture}
    \begin{axis}[
        xlabel={Число потоков/процессов},
        ylabel={Время (мс)},
        legend pos=north east,
    ]
    \addplot table [x index = 0, y index = 3, col sep=comma] {data/omp.csv};
    \addplot table [x index = 0, y index = 3, col sep=comma] {data/mpi.csv};
    \legend{\textit{OpenMP}, \textit{MPI}}
    \end{axis}
\end{tikzpicture}

\vspace{0.5cm}

При 6-ти потоках их производительность в данной задаче можно считать одинаковой.
Однако разработка программы на \textit{MPI} заняла гораздо больше времени и ресурсов.
