\appendix

\titleformat{\section}[display]
  {\normalfont\Large\bfseries}
  {\centering Приложение\ \thesection\\}
  {0pt}{\Large\centering}
\renewcommand{\thesection}{\Asbuk{section}}

\section{Использованные программные коды}

Для проверки версии \textit{OpenMP} использовался следующий код:
\lstinputlisting[language=C, basicstyle=\small]{code/omp_ver.c}
\vspace{0.5cm}

Для измерения времени исполнения программы с использованием \textit{OpenMP} использовался следующий код(выводит \textit{csv} в стандартный вывод):
\lstinputlisting[language=C, basicstyle=\tiny]{code/omp_main.c}
\vspace{0.5cm}

Для измерения времени исполнения программы с использованием \textit{MPI} использовался следующий код(выводит \textit{csv} в стандартный вывод):
\lstinputlisting[language=C, basicstyle=\tiny]{code/mpi_main.c}
\vspace{0.5cm}

А так же для этой цели использовался скрипт:
\lstinputlisting[language=Python, basicstyle=\tiny]{code/runmpi.py}
\vspace{0.5cm}

\section{Таблицы c практическими результатами}

\textit{OpenMP}:

\vspace{0.3cm}

\pgfplotstabletypeset[
 col sep=comma,
 columns={Threads,Worst (ms),Best (ms),Avg (ms)},
]{data/omp.csv}

\vspace{0.5cm}

\textit{MPI}:

\vspace{0.3cm}

\pgfplotstabletypeset[
 col sep=comma,
 columns={Threads,Worst (ms),Best (ms),Avg (ms)},
]{data/mpi.csv}

\vspace{0.5cm}
