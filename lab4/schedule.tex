\section{Применение \textit{schedule}}

За основу был взят код из второй лабораторной.
Число запусков на поток было уменьшено до 10.
Использовалось \textit{schedule(runtime)}, которое затем менялось в различных тестах.

Использованные исходные коды можно найти в приложении.

\subsection{Графики}

Для \textit{schedule = static}:

\vspace{0.3cm}

\begin{tikzpicture}
 \begin{axis}[
    xlabel={Число потоков},
    ylabel={Время (мс)},
    legend pos=north east,
  ]
  \addplot table [x index=0, y index=1, col sep=comma] {data/static.csv};
  \addplot table [x index=0, y index=1, col sep=comma] {data/static_10.csv};
  \addplot table [x index=0, y index=1, col sep=comma] {data/static_10000.csv};
  \addplot table [x index=0, y index=1, col sep=comma] {data/static_10000000.csv};
  \legend{Размер чанка не задан, Размер чанка 10, Размер чанка 10000, Размер чанка 10000000}
 \end{axis}
\end{tikzpicture}

\vspace{0.5cm}

Замечу, что компилятор сам выбирает оптимальное значение чанка.
Его производительность является одной из лучших.

Рассмотрим \textit{schedule = dynamic}:

\vspace{0.3cm}

\begin{tikzpicture}
 \begin{axis}[
    xlabel={Число потоков},
    ylabel={Время (мс)},
    legend pos=north east,
  ]
  \addplot table [x index=0, y index=1, col sep=comma] {data/dynamic.csv};
  \addplot table [x index=0, y index=1, col sep=comma] {data/dynamic_10.csv};
  \addplot table [x index=0, y index=1, col sep=comma] {data/dynamic_10000.csv};
  \addplot table [x index=0, y index=1, col sep=comma] {data/dynamic_10000000.csv};
  \legend{Размер чанка не задан, Размер чанка 10, Размер чанка 10000, Размер чанка 10000000}
 \end{axis}
\end{tikzpicture}

\vspace{0.5cm}

На удивление здесь стандартное значение показывает себя худшим образом.
По умолчанию в режиме \textit{dynamic} размер чанка равен 1.

Рассмотрим \textit{schedule = guided}:

\vspace{0.3cm}

\begin{tikzpicture}
 \begin{axis}[
    xlabel={Число потоков},
    ylabel={Время (мс)},
    legend pos=north east,
  ]
  \addplot table [x index=0, y index=1, col sep=comma] {data/guided.csv};
  \addplot table [x index=0, y index=1, col sep=comma] {data/guided_10.csv};
  \addplot table [x index=0, y index=1, col sep=comma] {data/guided_10000.csv};
  \addplot table [x index=0, y index=1, col sep=comma] {data/guided_10000000.csv};
  \legend{Размер чанка не задан, Размер чанка 10, Размер чанка 10000, Размер чанка 10000000}
 \end{axis}
\end{tikzpicture}

\vspace{0.5cm}

Здесь тенденция наблюдается положительная, за исключением чанков максимального размера.
Разница между остальными размерами чанков не превышает нескольких миллисекунд. 

Наконец, сравним лучшие из вышеприведенных результатов в автоматическим распределением \textit{schedule = auto}.
На графике указан режим распределения и размер чанка (если он был задан вручную):

\vspace{0.3cm}

\begin{tikzpicture}
 \begin{axis}[
    xlabel={Число потоков},
    ylabel={Время (мс)},
    legend pos=north east,
  ]
  \addplot table [x index=0, y index=1, col sep=comma] {data/static.csv};
  \addplot table [x index=0, y index=1, col sep=comma] {data/dynamic_10000.csv};
  \addplot table [x index=0, y index=1, col sep=comma] {data/guided.csv};
  \addplot table [x index=0, y index=1, col sep=comma] {data/auto.csv};
  \legend{\textit{static}, \textit{dynamic 10000}, \textit{guided}, \textit{auto}}
 \end{axis}
\end{tikzpicture}

\vspace{0.5cm}
