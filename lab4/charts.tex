\section{Экспериментальные данные}

На каждое число потоков отводилось 10 запусков.
Так же число элементов в массиве было уменьшено с \textit{10000000} до \textit{1000000}.

\subsection{Время выполнения}

Для начала я решил взглянуть не только на среднюю скорость выполнения, но и на крайние варианты:

\vspace{0.5cm}

\begin{tikzpicture}
    \begin{axis}[
        xlabel={Число потоков},
        ylabel={Время (мс)},
        legend pos=north east,
    ]
    \addplot table [x index = 0, y index = 1, col sep=comma] {data/base.csv};
    \addplot table [x index = 0, y index = 2, col sep=comma] {data/base.csv};
    \addplot table [x index = 0, y index = 3, col sep=comma] {data/base.csv};
    \legend{Худшее время, Лучшее время, Среднее время}
    \end{axis}
\end{tikzpicture}

\vspace{0.5cm}

Крайне заметно, что однопотчная программа работает куда быстрее её конкурентов.
В этой лабораторной худшие запуски занимали почти секунду, так что причины такой ``антипроизводительности'' остаются для меня загадкой.
Возможно причина кроется в распределении итераций цикла -- но это тема для одной из следующих лабораторных работ.

Рассмотрим теперь данные с оптимизацией:

\vspace{0.5cm}

\begin{tikzpicture}
    \begin{axis}[
        xlabel={Число потоков},
        ylabel={Время (мс)},
        legend pos=north east,
    ]
    \addplot table [x index = 0, y index = 1, col sep=comma] {data/opt.csv};
    \addplot table [x index = 0, y index = 2, col sep=comma] {data/opt.csv};
    \addplot table [x index = 0, y index = 3, col sep=comma] {data/opt.csv};
    \legend{Худшее время, Лучшее время, Среднее время}
    \end{axis}
\end{tikzpicture}

\vspace{0.5cm}

Как видно на графике выше, повышение числа потоков лишь увеличивает среднее время исполнения.

\subsection{Прирост производительности}

В целом с увеличением числа потоков производительность падает.
Рассмотрим ускорение многопоточной программы отностиельно однопоточной.
Для неоптимизированной сборки:

\vspace{0.5cm}

\begin{tikzpicture}
    \begin{axis}[
        xlabel={Число потоков},
        ylabel={Прирост (\%)},
        ybar interval=0.7,
    ]
    \addplot table [x index = 0, y index = 1, col sep=comma] {data/cmp_base.csv};
    \end{axis}
\end{tikzpicture}

\vspace{0.5cm}

Для оптимизированной сборки:

\vspace{0.5cm}

\begin{tikzpicture}
    \begin{axis}[
        xlabel={Число потоков},
        ylabel={Прирост (\%)},
        ybar interval=0.7,
    ]
    \addplot table [x index = 0, y index = 1, col sep=comma] {data/cmp_opt.csv};
    \end{axis}
\end{tikzpicture}

\vspace{0.5cm}
