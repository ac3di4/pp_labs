\documentclass[a4paper, 12pt]{article}

\usepackage[T2A]{fontenc}
\usepackage[utf8]{inputenc}
\usepackage[russian]{babel}
\usepackage{geometry}
\usepackage{setspace}
\usepackage{hyperref}
\usepackage{listings}
\usepackage{graphicx}
\usepackage{indentfirst}
\usepackage{titlesec}
\usepackage{pgfplots}
\usepackage{pgfplotstable}

\pgfplotsset{width=15cm, compat=1.18}
\newgeometry{left=1.5 cm, right=1.5cm, top=1.5cm, bottom=1.5cm}

\begin{document}


% ---------------------------------- Титульник ----------------------------------
\hypersetup{pageanchor=false}
\begin{titlepage}
 \begin{center}
  \vspace*{1cm}

  \Huge
  \textbf{Лабораторная работа №4}

  \vspace{0.5cm}
  \LARGE
  ``Технология OpenMP. Особенности настройки''

  \vspace{1.5cm}
  Выполнил студент группы Б20-505\\
  \textbf{Сорочан Илья}

  \vfill

  \Large
  Московский Инженерно-Физический Интститут\\
  Москва 2023

 \end{center}
\end{titlepage}


% ---------------------------------- Рабочая среда ----------------------------------

\section{Рабочая среда}

Технические характеристики (вывод \textit{inxi}):
\begin{verbatim}
CPU: 6-core AMD Ryzen 5 4500U with Radeon Graphics (-MCP-)
speed/min/max: 1396/1400/2375 MHz Kernel: 5.15.85-1-MANJARO x86_64 Up: 46m
Mem: 2689.5/7303.9 MiB (36.8%) Storage: 238.47 GiB (12.6% used) Procs: 238
Shell: Zsh inxi: 3.3.24
\end{verbatim}

Используемый компилятор:
\begin{verbatim}
gcc (GCC) 12.2.0
\end{verbatim}

Согласно \href{https://www.openmp.org/resources/openmp-compilers-tools/}{официальной документации} даная версия компилятора поддерживает \textit{OpenMP 5.0}


% ---------------------------------- Работа с OpenMP ----------------------------------

\section{Работа с \textit{OpenMP}}

\subsection{Версия и дата принятия}

Макрос \textit{\_OPENMP} является целочисленным числом и показывает дату принятия \textit{OpenMP} в формате \textit{yyyymm}, где \textit{yyyy} - год принятия, а \textit{mm} - месяц.

Даты можно посмотреть на \href{https://www.openmp.org/specifications/}{официальном сайте \textit{OpenMP}}. но в своем коде я написал удобный макрос.

\subsection{\textit{OMP\_DYNAMIC}}

Переменная окружения \textit{OMP\_DYNAMIC} отвечает за динамический выбор числа потоков. Например если она имеет значение \textit{true}, то \textit{OpenMP} автоматически выбирает число потоков для \textit{parallel} участков. Если же \textit{false},

\subsection{wtick}

Функция \textit{omp\_get\_wtick()} возвращает количество секунд, прошедшее между тиками таймера из \textit{omp\_get\_wtime()}

\subsection{Вложенность}

Функция \textit{omp\_get\_nested()} возвращает флаг, указывающий на то включен ли вложеннный параллелилизм. Если да, то количество вложенных конструкций ограниченно числом, которое можно получить, вызвав \textit{omp\_get\_max\_active\_levels()}.

\subsection{schedule}

Переменная окружения \textit{OMP\_SCHEDULE} задаёт тип распределения нагрузки и размер чанков для всех директив циклов. Тип определяет как циклы делятся на подмножества итераций размером в один чанк:
\begin{itemize}
 \item \textit{static} -- все подмножества распределяются между потоками один раз, в самом начале;
 \item \textit{dynamic} -- каждый из процессов получает чанк, по его выполнении он запрашивает новый. Так продолжается пока чанки не закончатся;
 \item \textit{guided} -- аналогично \textit{dynamic}, однако он не содержит чанка, размер которого меньше заданного размера чанка;
 \item \textit{auto} -- компилятор выбирает на свое усмотрение;
 \item \textit{runtime} -- выбор производится непосредственно перед выполнением цикла.
\end{itemize}

\subsection{Пример использования \textit{omp\_lock}}

Замки необходимы для обеспечения выболнения промежутка кода только одним потоком.
Например чтение из файла.

\begin{lstlisting}[language=C]
omp_lock_t writelock;

omp_init_lock(&writelock);

#pragma omp parallel for
for ( i = 0; i < x; i++ )
{
  // do something important
  omp_set_lock(&writelock);
  // do something important but only one thread access at a time
  omp_unset_lock(&writelock);
  // do another important task
}

omp_destroy_lock(&writelock);
\end{lstlisting}

\subsection{Разработанный код}

Для иллюстрации директив \textit{OpenMP}, затронутых в данном разделе была разработна следующая программа:
\lstinputlisting[language=C, basicstyle=\tiny]{src/omp_info.c}
\vspace{0.5cm}


% ---------------------------------- schedule ----------------------------------

\section{Применение schedule}

\subsection{Исходный код}

В качестве примера я взял свой код из третьей лабораторной работы, однако переработал его, что бы при компиляции можно было указывать не только число потоков, но и расписание вместе с размером чанка:

\lstinputlisting[language=C, basicstyle=\tiny]{src/main.c}
\vspace{0.5cm}

Скрипт так же был изменен:

\lstinputlisting[language=Python, basicstyle=\tiny]{src/main.py}
\vspace{0.5cm}

Так же в этот раз я делал не по 10, а по 5 запусков на поток. Это связано с большим измеряемым объемом данных.

\subsection{Графики}

Для $schedule = static$:

\vspace{0.3cm}

\begin{tikzpicture}
 \begin{axis}[
    xlabel={Число потоков},
    ylabel={Время (мс)},
    legend pos=north east,
  ]
  \addplot table [x index=0, y index=1, col sep=comma] {data/static.csv};
  \addplot table [x index=0, y index=1, col sep=comma] {data/static_1.csv};
  \addplot table [x index=0, y index=1, col sep=comma] {data/static_100.csv};
  \addplot table [x index=0, y index=1, col sep=comma] {data/static_10000.csv};
  \addplot table [x index=0, y index=1, col sep=comma] {data/static_100000.csv};
  \legend{Размер чанка не задан, Размер чанка 1, Размер чанка 100, Размер чанка 10000, Размер чанка 100000}
 \end{axis}
\end{tikzpicture}

\vspace{0.5cm}

Для $schedule = dynamic$:

\vspace{0.3cm}

\begin{tikzpicture}
 \begin{axis}[
    xlabel={Число потоков},
    ylabel={Время (мс)},
    legend pos=north east,
  ]
  \addplot table [x index=0, y index=1, col sep=comma] {data/dynamic.csv};
  \addplot table [x index=0, y index=1, col sep=comma] {data/dynamic_1.csv};
  \addplot table [x index=0, y index=1, col sep=comma] {data/dynamic_100.csv};
  \addplot table [x index=0, y index=1, col sep=comma] {data/dynamic_10000.csv};
  \addplot table [x index=0, y index=1, col sep=comma] {data/dynamic_100000.csv};
  \legend{Размер чанка не задан, Размер чанка 1, Размер чанка 100, Размер чанка 10000, Размер чанка 100000}
 \end{axis}
\end{tikzpicture}

\vspace{0.5cm}


Для $schedule = guided$:

\vspace{0.3cm}

\begin{tikzpicture}
 \begin{axis}[
    xlabel={Число потоков},
    ylabel={Время (мс)},
    legend pos=north east,
  ]
  \addplot table [x index=0, y index=1, col sep=comma] {data/guided.csv};
  \addplot table [x index=0, y index=1, col sep=comma] {data/guided_1.csv};
  \addplot table [x index=0, y index=1, col sep=comma] {data/guided_100.csv};
  \addplot table [x index=0, y index=1, col sep=comma] {data/guided_10000.csv};
  \addplot table [x index=0, y index=1, col sep=comma] {data/guided_100000.csv};
  \legend{Размер чанка не задан, Размер чанка 1, Размер чанка 100, Размер чанка 10000, Размер чанка 100000}
 \end{axis}
\end{tikzpicture}

\vspace{0.3cm}

Из графиков видно, что значения по умолчанию являются оптимальными. Для остальных режимов (\textit{auto} и \textit{runtime}) чанки не указывались. Рассмотрим сравнение этих режимов:

\vspace{0.3cm}

\begin{tikzpicture}
 \begin{axis}[
    xlabel={Число потоков},
    ylabel={Время (мс)},
    legend pos=north east,
  ]
  \addplot table [x index=0, y index=1, col sep=comma] {data/static.csv};
  \addplot table [x index=0, y index=1, col sep=comma] {data/dynamic.csv};
  \addplot table [x index=0, y index=1, col sep=comma] {data/guided.csv};
  \addplot table [x index=0, y index=1, col sep=comma] {data/auto.csv};
  \addplot table [x index=0, y index=1, col sep=comma] {data/runtime.csv};
  \legend{Режим \textit{static}, Режим \textit{dynamic}, Режим \textit{guided}, Режим \textit{auto}, Режим \textit{runtime}}
 \end{axis}
\end{tikzpicture}


% ---------------------------------- Заключение ----------------------------------

\section{Заключение}
В данной работы были исследованы некоторые директивы \textit{OpenMP}. Была усовершенствована программа из третьей лабораторной работы. Модифицирован скрипт и произведены замеры с различными распределениями нагрузки в цикле.

В ходе работы было выяснено, что тип распределения \textit{static} куда быстрее \textit{dynamic}.

Хочу заметить, что при использовании \textit{static} многопоточная программа без оптимизаций выполняется гораздо быстрее однопоточной программы с ними (результаты в предыдущей лабораторной). Это говорит о том, что результаты многопоточных программ в третьей лабораторной работе могут быть улучшены.


\end{document}
