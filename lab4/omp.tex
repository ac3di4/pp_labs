\section{Работа с \textit{OpenMP}}

\subsection{Версия и дата принятия}

Макрос \textit{\_OPENMP} является целочисленным числом и показывает дату принятия \textit{OpenMP} в формате \textit{yyyymm}, где \textit{yyyy} - год принятия, а \textit{mm} - месяц.
Соответствие даты версии \textit{OpenMP} можно найти в интернете.

\subsection{\textit{OMP\_DYNAMIC}}

Переменная окружения \textit{OMP\_DYNAMIC} отвечает за динамический выбор числа потоков. Например если она имеет значение \textit{true}, то \textit{OpenMP} автоматически выбирает число потоков для \textit{parallel} участков.
Если же \textit{false},

\subsection{wtick}

Функция \textit{omp\_get\_wtick()} возвращает количество секунд, прошедшее между тиками таймера из \textit{omp\_get\_wtime()}.
\textit{wtime} возвращает количество прошедших секунд.

Стоит отметить, что функция \textit{clock()}, предоставленная в стандартной библиотеке \textit{С} считает время использования \textit{CPU}.
Тогда как \textit{omp\_wtime()} считает просто прошедшее время. 

\subsection{Вложенность}

Функция \textit{omp\_get\_nested()} возвращает флаг, указывающий на то включен ли вложенный параллелизм.
Если он истинен, то количество вложенных конструкций ограниченно числом, которое можно получить, вызвав \textit{omp\_get\_max\_active\_levels()}.

\subsection{schedule}

Опция \textit{schedule} директивы \textit{for} задаёт тип распределения нагрузки и размер чанков. Тип определяет как циклы делятся на подмножества итераций -- чанки:
\begin{itemize}
 \item \textit{static} -- все подмножества распределяются между потоками один раз, в самом начале;
 \item \textit{dynamic} -- каждый из процессов получает чанк, по его выполнении он запрашивает новый. Так продолжается пока чанки не закончатся;
 \item \textit{guided} -- аналогично \textit{dynamic}, однако по мере выполнения чанков их размер уменьшается;
 \item \textit{auto} -- компилятор выбирает на свое усмотрение;
 \item \textit{runtime} -- выбор производится непосредственно перед выполнением программы через переменную окружения \textit{OMP\_SCHEDULE}.
\end{itemize}

\subsection{Пример использования \textit{omp\_lock}}

Замки необходимы для обеспечения выполнении промежутка кода только одним потоком.
Например чтение из файла.

\begin{lstlisting}[language=C]
omp_lock_t write_lock;

omp_init_lock(&write_lock);

#pragma omp parallel for
for ( i = 0; i < x; i++ )
{
  // do something important
  omp_set_lock(&write_lock);
  // do something important but only one thread access at a time
  omp_unset_lock(&write_lock);
  // do another important task
}

omp_destroy_lock(&write_lock);
\end{lstlisting}

\subsection{Разработанный код}

Для иллюстрации директив \textit{OpenMP}, затронутых в данном разделе была разработана следующая программа:
\lstinputlisting[language=C, basicstyle=\tiny]{code/omp_info.c}
\vspace{0.5cm}
