\documentclass[a4paper, 12pt]{article}

\usepackage{prelab}

\usepackage{graphicx}
\usepackage{pgfplots}
\usepackage{pgfplotstable}
\usepackage{titlesec}
\usepackage{listings}

\pgfplotsset{width=12cm, compat=1.18}

\begin{document}

\prefixlab{2}{Выделение ресурса параллелизма. Технология \textit{OpenMP}}

\section{Анализ алгоритма}

\subsection{Принцип работы}

Алгоритм является поиском максимума в массиве.
Программа:
\begin{enumerate}
    \item Выделяет память под массив, инициализирует генератор случайных значений;
    \item Заполняет массив случайными числами;
    \item Ищет максимум с определенными настройками \textit{OpenMP}.
\end{enumerate}

Поиск максимума осуществляется параллельно. 
Массив разбивается на секции (почти) равного размера, которые распределяются между потоками.
Распределение зависит от параметра \textit{schedule}, который здесь не указан.
В таком случае он определяется компилятором/ОС.

Временная сложность алгоритма $O(\frac{n}{p})$, где:
\begin{itemize}
 \item $n$ -- число элементов в массиве;
 \item $p$ -- число используемых потоков.
\end{itemize}

Блок-схема алгоритма в одном потоке выглядит следующим образом:\\
\includegraphics[scale=0.6]{res/flowchart.png}

\subsection{Директивы \textit{OpenMP}}

Поясним представленные директивы \textit{OpenMP}.

Директива \textit{parallel} задает опции параллелизации:
\begin{itemize}
    \item \textit{num\_threads} -- число потоков;
    \item \textit{shared} -- общая для потоков память;
    \item \textit{reduction} -- способ объединения локальных переменных в глобальную. В данном случае вычисление максимума;
    \item \textit{default} -- локальность переменных \textit{по умолчанию}. В данном случае все переменные по умолчанию локальные.
\end{itemize}
Если бы данной директивы не было, то следующий за ней блок кода исполнялся бы одним потоком без участия \textit{OpenMP}. 

Директива \textit{for} используя опции, задаваемые директивой \textit{parallel} распределяет итерации цикла между потоками.
Если бы данной директивы не было, то цикл, следующий за ней, выполнился бы во всех потоках (не было бы распределения итераций).

\section{Экспериментальные данные}

На каждое число потоков отводилось 20 запусков.

\subsection{Время выполнения}

Для начала я решил взглянуть не только на среднюю скорость выполнения, но и на крайние варианты:

\vspace{0.5cm}

\begin{tikzpicture}
    \begin{axis}[
        xlabel={Число потоков},
        ylabel={Время (мс)},
        legend pos=north east,
    ]
    \addplot table [x index = 0, y index = 1, col sep=comma] {data/base.csv};
    \addplot table [x index = 0, y index = 2, col sep=comma] {data/base.csv};
    \addplot table [x index = 0, y index = 3, col sep=comma] {data/base.csv};
    \legend{Худшее время, Лучшее время, Среднее время}
    \end{axis}
\end{tikzpicture}

\vspace{0.5cm}

Крайне заметно, что многопоточная программа работает куда быстрее её конкурентов.
Единственным исключением является запуск при 2-x потоках.

Аналогично первой лабораторной рассмотрим теперь данные с оптимизацией:

\vspace{0.5cm}

\begin{tikzpicture}
    \begin{axis}[
        xlabel={Число потоков},
        ylabel={Время (мс)},
        legend pos=north east,
    ]
    \addplot table [x index = 0, y index = 1, col sep=comma] {data/opt.csv};
    \addplot table [x index = 0, y index = 2, col sep=comma] {data/opt.csv};
    \addplot table [x index = 0, y index = 3, col sep=comma] {data/opt.csv};
    \legend{Худшее время, Лучшее время, Среднее время}
    \end{axis}
\end{tikzpicture}

\vspace{0.5cm}

Как видно на графике выше, повышение числа потоков уменьшает среднее время исполнения.

\subsection{Прирост производительности}

В целом с увеличением числа потоков производительность растет.
Рассмотрим ускорение многопоточной программы относительно однопоточной.
Для не оптимизированной сборки:

\vspace{0.5cm}

\begin{tikzpicture}
    \begin{axis}[
        xlabel={Число потоков},
        ylabel={Прирост (\%)},
        ybar interval=0.7,
    ]
    \addplot table [x index = 0, y index = 1, col sep=comma] {data/cmp_base.csv};
    \end{axis}
\end{tikzpicture}

\vspace{0.5cm}

Для оптимизированной сборки:

\vspace{0.5cm}

\begin{tikzpicture}
    \begin{axis}[
        xlabel={Число потоков},
        ylabel={Прирост (\%)},
        ybar interval=0.7,
    ]
    \addplot table [x index = 0, y index = 1, col sep=comma] {data/cmp_opt.csv};
    \end{axis}
\end{tikzpicture}

\vspace{0.5cm}


\section{Заключение}

В данной работе было исследовано ускорение, получаемое при использовании нескольких потоков в задании о поиске элемента.
Была усовершенствована предоставленная программа и собранны данные.
Так же был написан скрипт, подсчитывающий прирост производительности относительно одного потока.
Оформлен отчет.

В ходе работы было выяснено, что в применение нескольких потоков крайне положительно влияет на итоговую производительность.
Из 30 многопоточных сборок только одна была медленнее однопоточной.
При этом наблюдался прирост вплоть до 2-х с половиной раз.

\appendix

\titleformat{\section}[display]
  {\normalfont\Large\bfseries}
  {\centering Приложение\ \thesection\\}
  {0pt}{\Large\centering}
\renewcommand{\thesection}{\Asbuk{section}}

\section{Использованные программные коды}

Для проверки версии \textit{OpenMP} использовался следующий код:
\lstinputlisting[language=C, basicstyle=\small]{code/omp_ver.c}
\vspace{0.5cm}

Для измерения времени исполнения алгоритма использовался следующий код (выводит \textit{csv} в стандартный вывод):
\lstinputlisting[language=C, basicstyle=\tiny]{code/main.c}
\vspace{0.5cm}

Для вычисления эффективности многопоточной программы по отношению к однопоточной использовался следующий скрипт:
\lstinputlisting[language=Python, basicstyle=\small]{code/compare.py}
\vspace{0.5cm}

\section{Таблицы c практическими результатами}

Таблица без оптимизаций:

\vspace{0.3cm}

\pgfplotstabletypeset[
 col sep=comma,
 columns={Threads,Worst (ms),Best (ms),Avg (ms)},
]{data/base.csv}

\vspace{0.5cm}

Таблица с оптимизациями:

\vspace{0.3cm}

\pgfplotstabletypeset[
 col sep=comma,
 columns={Threads,Worst (ms),Best (ms),Avg (ms)},
]{data/opt.csv}

\vspace{0.5cm}

Таблица сравнений без оптимизаций:

\vspace{0.3cm}

\pgfplotstabletypeset[
 col sep=comma,
 columns={Threads,Efficiency},
]{data/cmp_base.csv}

\vspace{0.5cm}

Таблица сравнений c оптимизациями:

\vspace{0.3cm}

\pgfplotstabletypeset[
 col sep=comma,
 columns={Threads,Efficiency},
]{data/cmp_opt.csv}


\end{document}