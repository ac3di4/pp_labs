\pgfplotsset{width=10cm, compat=1.18}

\section{Экспериментальные данные}

\subsection{Модификации кода}

Исходный код программы был модифицирован так, что бы:
\begin{itemize}
    \item Освобождалась \textbf{вся} выделенная память;
    \item Производился запуск с несколькими потоками;
    \item Замерялось время, затраченное на выполнение алгоритма.
\end{itemize}

На каждое число потоков отводилось 20 запусков.

\subsection{Количество операций сравнения}

В измерении количества операций сравнения нет смысла, ведь для однопоточной программы:
\textit{O(n)}, как и для многопоточной (сравнения из \textit{reduction} -- сравнения с крайними элементами секций).

\subsection{Время выполнения}

Для начала я решил взглянуть не только на среднюю скорость выполнения, но и на крайние варианты:

\vspace{0.5cm}

\begin{tikzpicture}
    \begin{axis}[
        xlabel={Число потоков},
        ylabel={Время (мс)},
        legend pos=north east,
    ]
    \addplot table [x index = 0, y index = 1, col sep=comma] {data/base.csv};
    \addplot table [x index = 0, y index = 2, col sep=comma] {data/base.csv};
    \addplot table [x index = 0, y index = 3, col sep=comma] {data/base.csv};
    \legend{Худшее время, Лучшее время, Среднее время}
    \end{axis}
\end{tikzpicture}

\vspace{0.5cm}

Многопоточная программа практически во всех случаях работает гораздо быстрее.
Исключением стали 2 потока, но небольшая разница в производительности позволяет списать это на погрешность.
Рассмотрим как изменится ситуация при включении оптимизаций:

\vspace{0.5cm}

\begin{tikzpicture}
    \begin{axis}[
        xlabel={Число потоков},
        ylabel={Время (мс)},
        legend pos=north east,
    ]
    \addplot table [x index = 0, y index = 1, col sep=comma] {data/opt.csv};
    \addplot table [x index = 0, y index = 2, col sep=comma] {data/opt.csv};
    \addplot table [x index = 0, y index = 3, col sep=comma] {data/opt.csv};
    \legend{Худшее время, Лучшее время, Среднее время}
    \end{axis}
\end{tikzpicture}

\vspace{0.5cm}

Как видно на графике выше, общая тенденция остается неизменной.
Оптимизации, производимые компилятором положительно влияют на производительность вне зависимости от числа используемых потоков: 

\vspace{0.5cm}

\begin{tikzpicture}
    \begin{axis}[
        xlabel={Число потоков},
        ylabel={Время (мс)},
        legend pos=north east,
    ]
    \addplot table [x index = 0, y index = 3, col sep=comma] {data/base.csv};
    \addplot table [x index = 0, y index = 3, col sep=comma] {data/opt.csv};
    \legend{Без оптимизации, С оптимизацией}
    \end{axis}
\end{tikzpicture}

\vspace{0.5cm}

\subsection{Прирост производительности}

Как уже было замечено ранее, в целом с увеличением числа потоков производительность растет.
Рассмотрим ускорение многопоточной программы относительно однопоточной.
Для не оптимизированной сборки:

\vspace{0.5cm}

\begin{tikzpicture}
    \begin{axis}[
        xlabel={Число потоков},
        ylabel={Прирост (\%)},
        ybar interval=0.7,
    ]
    \addplot table [x index = 0, y index = 1, col sep=comma] {data/cmp_base.csv};
    \end{axis}
\end{tikzpicture}

\vspace{0.5cm}

Для оптимизированной сборки:

\vspace{0.5cm}

\begin{tikzpicture}
    \begin{axis}[
        xlabel={Число потоков},
        ylabel={Прирост (\%)},
        ybar interval=0.7,
    ]
    \addplot table [x index = 0, y index = 1, col sep=comma] {data/cmp_opt.csv};
    \end{axis}
\end{tikzpicture}

\vspace{0.5cm}

Наибольший прирост наблюдается при отсутствии оптимизаций и практически достигает 300\%!
